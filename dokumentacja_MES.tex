\documentclass[12pt,a4paper]{article}
\usepackage[utf8]{inputenc}
\usepackage[T1]{fontenc}
\usepackage[polish]{babel}
\usepackage{amsmath}
\usepackage{amsfonts}
\usepackage{amssymb}
\usepackage{graphicx}
\usepackage{geometry}
\usepackage{listings}
\usepackage{xcolor}
\usepackage{hyperref}
\usepackage{float}
\usepackage{booktabs}
\usepackage{multirow}

\geometry{margin=2.5cm}

\title{Metoda Elementów Skończonych w Analizie Przewodzenia Ciepła\\
\large Dokumentacja Matematyczna i Implementacja}
\author{}
\date{\today}

\begin{document}

\maketitle
\tableofcontents
\newpage

\section{Wprowadzenie}

Niniejszy dokument przedstawia kompletną analizę matematyczną i implementacyjną programu opartego na Metodzie Elementów Skończonych (MES) dla rozwiązywania zagadnień przewodzenia ciepła w dwuwymiarowych obszarach. Program obsługuje zarówno analizę stacjonarną (stan ustalony), jak i niestacjonarną (zmienną w czasie).

\section{Podstawy Teoretyczne -- Funkcje Kształtu}

\subsection{Element Czterowęzłowy (Q4)}

Program wykorzystuje element czterowęzłowy (czworokątny) w układzie współrzędnych naturalnych $(\xi, \eta)$. Numeracja węzłów odbywa się przeciwnie do ruchu wskazówek zegara:

\begin{itemize}
    \item Węzeł 1: $(\xi, \eta) = (-1, -1)$
    \item Węzeł 2: $(\xi, \eta) = (1, -1)$
    \item Węzeł 3: $(\xi, \eta) = (1, 1)$
    \item Węzeł 4: $(\xi, \eta) = (-1, 1)$
\end{itemize}

\subsection{Funkcje Kształtu}

Funkcje kształtu dla elementu bilinearnego czterowęzłowego definiowane są wzorami:

\begin{equation}
N_1(\xi, \eta) = \frac{1}{4}(1-\xi)(1-\eta)
\end{equation}

\begin{equation}
N_2(\xi, \eta) = \frac{1}{4}(1+\xi)(1-\eta)
\end{equation}

\begin{equation}
N_3(\xi, \eta) = \frac{1}{4}(1+\xi)(1+\eta)
\end{equation}

\begin{equation}
N_4(\xi, \eta) = \frac{1}{4}(1-\xi)(1+\eta)
\end{equation}

\subsection{Własności Funkcji Kształtu}

Funkcje kształtu spełniają następujące właściwości:

\begin{itemize}
    \item \textbf{Warunek pełności (partition of unity):}
    \begin{equation}
    \sum_{i=1}^{4} N_i(\xi, \eta) = 1
    \end{equation}
    
    \item \textbf{Właściwość Kroneckera:}
    \begin{equation}
    N_i(\xi_j, \eta_j) = \delta_{ij} = \begin{cases} 
    1 & \text{jeśli } i = j \\
    0 & \text{jeśli } i \neq j
    \end{cases}
    \end{equation}
    
    \item \textbf{Interpolacja temperatury:}
    \begin{equation}
    T(\xi, \eta) = \sum_{i=1}^{4} N_i(\xi, \eta) \cdot T_i
    \end{equation}
\end{itemize}

\textbf{Pochodzenie:} Funkcje kształtu są iloczynem funkcji liniowych w kierunkach $\xi$ i $\eta$, co zapewnia ciągłość temperatury między elementami.

\section{Pochodne Funkcji Kształtu}

\subsection{Pochodne względem $\xi$}

\begin{equation}
\frac{\partial N_1}{\partial \xi} = -\frac{1}{4}(1-\eta)
\end{equation}

\begin{equation}
\frac{\partial N_2}{\partial \xi} = \frac{1}{4}(1-\eta)
\end{equation}

\begin{equation}
\frac{\partial N_3}{\partial \xi} = \frac{1}{4}(1+\eta)
\end{equation}

\begin{equation}
\frac{\partial N_4}{\partial \xi} = -\frac{1}{4}(1+\eta)
\end{equation}

\subsection{Pochodne względem $\eta$}

\begin{equation}
\frac{\partial N_1}{\partial \eta} = -\frac{1}{4}(1-\xi)
\end{equation}

\begin{equation}
\frac{\partial N_2}{\partial \eta} = -\frac{1}{4}(1+\xi)
\end{equation}

\begin{equation}
\frac{\partial N_3}{\partial \eta} = \frac{1}{4}(1+\xi)
\end{equation}

\begin{equation}
\frac{\partial N_4}{\partial \eta} = \frac{1}{4}(1-\xi)
\end{equation}

\textbf{Pochodzenie:} Standardowe różniczkowanie funkcji kształtu według współrzędnych naturalnych $\xi$ i $\eta$.

\section{Kwadratura Gaussa}

Całkowanie numeryczne w MES realizowane jest za pomocą kwadratury Gaussa-Legendre'a. Program obsługuje schematy 2, 3 i 4-punktowe.

\subsection{Formuła Ogólna}

Kwadratura Gaussa aproksymuje całkę na przedziale $[-1, 1]$:

\begin{equation}
\int_{-1}^{1} f(\xi) \, d\xi \approx \sum_{i=1}^{n} w_i \cdot f(\xi_i)
\end{equation}

gdzie $\xi_i$ to punkty całkowania, a $w_i$ to wagi.

\subsection{Kwadratura 2-punktowa}

Dokładna dla wielomianów stopnia $\leq 3$:

\begin{equation}
\xi_i = \pm \frac{1}{\sqrt{3}} = \pm 0.577350269...
\end{equation}

\begin{equation}
w_i = 1.0
\end{equation}

\subsection{Kwadratura 3-punktowa}

Dokładna dla wielomianów stopnia $\leq 5$:

\begin{equation}
\xi_i = \left\{-\sqrt{\frac{3}{5}}, \, 0, \, \sqrt{\frac{3}{5}}\right\} = \{-0.774597, \, 0, \, 0.774597\}
\end{equation}

\begin{equation}
w_i = \left\{\frac{5}{9}, \, \frac{8}{9}, \, \frac{5}{9}\right\} = \{0.555556, \, 0.888889, \, 0.555556\}
\end{equation}

\subsection{Kwadratura 4-punktowa}

Dokładna dla wielomianów stopnia $\leq 7$:

\begin{align}
p_1 &= \sqrt{\frac{3}{7} - \frac{2}{7}\sqrt{\frac{6}{5}}} \\
p_2 &= \sqrt{\frac{3}{7} + \frac{2}{7}\sqrt{\frac{6}{5}}}
\end{align}

\begin{align}
w_1 &= \frac{18 + \sqrt{30}}{36} \\
w_2 &= \frac{18 - \sqrt{30}}{36}
\end{align}

Punkty całkowania: $\xi_i = \{-p_2, -p_1, p_1, p_2\}$

Wagi: $w_i = \{w_2, w_1, w_1, w_2\}$

\subsection{Całkowanie 2D}

Dla całek dwuwymiarowych stosuje się iloczyn tensorowy:

\begin{equation}
\int_{-1}^{1} \int_{-1}^{1} f(\xi, \eta) \, d\xi \, d\eta \approx \sum_{i=1}^{n} \sum_{j=1}^{n} w_i \cdot w_j \cdot f(\xi_i, \eta_j)
\end{equation}

\textbf{Pochodzenie:} Punkty i wagi kwadratury Gaussa-Legendre'a są pierwiastkami wielomianów Legendre'a, zapewniając optymalną dokładność numeryczną.

\section{Transformacja Geometryczna -- Jakobian}

\subsection{Macierz Jakobianu}

Transformacja z układu naturalnego $(\xi, \eta)$ do układu fizycznego $(x, y)$ opisana jest przez macierz Jakobianu:

\begin{equation}
\mathbf{J} = \begin{bmatrix}
\frac{\partial x}{\partial \xi} & \frac{\partial y}{\partial \xi} \\[0.3em]
\frac{\partial x}{\partial \eta} & \frac{\partial y}{\partial \eta}
\end{bmatrix}
\end{equation}

Składowe macierzy obliczane są z zależności:

\begin{equation}
\frac{\partial x}{\partial \xi} = \sum_{i=1}^{4} \frac{\partial N_i}{\partial \xi} \cdot x_i
\end{equation}

\begin{equation}
\frac{\partial y}{\partial \xi} = \sum_{i=1}^{4} \frac{\partial N_i}{\partial \xi} \cdot y_i
\end{equation}

\begin{equation}
\frac{\partial x}{\partial \eta} = \sum_{i=1}^{4} \frac{\partial N_i}{\partial \eta} \cdot x_i
\end{equation}

\begin{equation}
\frac{\partial y}{\partial \eta} = \sum_{i=1}^{4} \frac{\partial N_i}{\partial \eta} \cdot y_i
\end{equation}

\subsection{Wyznacznik Jakobianu}

\begin{equation}
\det(\mathbf{J}) = J_{11} \cdot J_{22} - J_{12} \cdot J_{21}
\end{equation}

\textbf{Znaczenie fizyczne:} $\det(\mathbf{J})$ reprezentuje współczynnik skalowania elementu powierzchni podczas transformacji. Musi być dodatni dla prawidłowego elementu (element niezdegenerowany).

\subsection{Odwrotna Macierz Jakobianu}

\begin{equation}
\mathbf{J}^{-1} = \frac{1}{\det(\mathbf{J})} \begin{bmatrix}
J_{22} & -J_{12} \\
-J_{21} & J_{11}
\end{bmatrix}
\end{equation}

\textbf{Pochodzenie:} Standardowa transformacja geometryczna w MES, umożliwiająca przeliczanie pochodnych między układami współrzędnych.

\section{Pochodne w Układzie Fizycznym}

Za pomocą odwrotnej macierzy Jakobianu transformujemy pochodne z układu naturalnego do fizycznego:

\begin{equation}
\begin{bmatrix}
\frac{\partial N_i}{\partial x} \\[0.3em]
\frac{\partial N_i}{\partial y}
\end{bmatrix}
= \mathbf{J}^{-1} \cdot
\begin{bmatrix}
\frac{\partial N_i}{\partial \xi} \\[0.3em]
\frac{\partial N_i}{\partial \eta}
\end{bmatrix}
\end{equation}

Jawnie:

\begin{equation}
\frac{\partial N_i}{\partial x} = J^{-1}_{11} \cdot \frac{\partial N_i}{\partial \xi} + J^{-1}_{12} \cdot \frac{\partial N_i}{\partial \eta}
\end{equation}

\begin{equation}
\frac{\partial N_i}{\partial y} = J^{-1}_{21} \cdot \frac{\partial N_i}{\partial \xi} + J^{-1}_{22} \cdot \frac{\partial N_i}{\partial \eta}
\end{equation}

\textbf{Pochodzenie:} Reguła łańcuchowa różniczkowania:
\begin{equation}
\frac{\partial}{\partial x} = \frac{\partial \xi}{\partial x} \cdot \frac{\partial}{\partial \xi} + \frac{\partial \eta}{\partial x} \cdot \frac{\partial}{\partial \eta}
\end{equation}

\section{Macierz Przewodnictwa H (Stiffness Matrix)}

\subsection{Teoria}

Macierz $\mathbf{H}$ reprezentuje przewodnictwo cieplne wewnątrz elementu. Wynika z równania Fouriera dla ustalonego przewodzenia ciepła.

\subsection{Wzór Całkowy}

\begin{equation}
H_{ij} = \int\int_{\Omega} k \left( \frac{\partial N_i}{\partial x} \cdot \frac{\partial N_j}{\partial x} + \frac{\partial N_i}{\partial y} \cdot \frac{\partial N_j}{\partial y} \right) \, d\Omega
\end{equation}

gdzie:
\begin{itemize}
    \item $k$ -- przewodność cieplna [W/(m$\cdot$K)]
    \item $\Omega$ -- obszar elementu
    \item $N_i, N_j$ -- funkcje kształtu
\end{itemize}

\subsection{Przykład Analityczny -- Prosty Element Kwadratowy}

Rozważmy prosty element kwadratowy o wymiarach $1 \times 1$ m, z węzłami w punktach:
\begin{itemize}
    \item Węzeł 1: $(0, 0)$
    \item Węzeł 2: $(1, 0)$
    \item Węzeł 3: $(1, 1)$
    \item Węzeł 4: $(0, 1)$
\end{itemize}

Dla tego elementu macierz Jakobianu jest stała (niezależna od $\xi, \eta$):

\begin{equation}
\mathbf{J} = \begin{bmatrix}
0.5 & 0 \\
0 & 0.5
\end{bmatrix}, \quad \det(\mathbf{J}) = 0.25, \quad \mathbf{J}^{-1} = \begin{bmatrix}
2 & 0 \\
0 & 2
\end{bmatrix}
\end{equation}

Dla punktu całkowania $(\xi, \eta) = (-1/\sqrt{3}, -1/\sqrt{3})$ w 2-punktowej kwadraturze Gaussa:

Pochodne funkcji kształtu w układzie naturalnym:
\begin{align}
\frac{\partial N_1}{\partial \xi} &= -0.25(1-\eta) = -0.3943 \\
\frac{\partial N_1}{\partial \eta} &= -0.25(1-\xi) = -0.3943
\end{align}

Pochodne w układzie fizycznym:
\begin{align}
\frac{\partial N_1}{\partial x} &= 2 \cdot (-0.3943) + 0 = -0.7886 \\
\frac{\partial N_1}{\partial y} &= 0 + 2 \cdot (-0.3943) = -0.7886
\end{align}

Wkład do elementu $H_{11}$ w tym punkcie całkowania (dla $k = 25$ W/(m$\cdot$K)):
\begin{equation}
H_{11} \mathrel{+}= 25 \cdot [(-0.7886)^2 + (-0.7886)^2] \cdot 0.25 \cdot 1.0 = 7.7825
\end{equation}

Po zsumowaniu wkładów ze wszystkich 4 punktów całkowania, otrzymujemy:
\begin{equation}
H_{11} \approx 31.13 \text{ W/K}
\end{equation}

\subsection{Aproksymacja Kwadraturą Gaussa}

W układzie naturalnych współrzędnych:

\begin{equation}
H_{ij} \approx \sum_{m=1}^{n_\xi} \sum_{n=1}^{n_\eta} w_m \cdot w_n \cdot k \left( \frac{\partial N_i}{\partial x} \cdot \frac{\partial N_j}{\partial x} + \frac{\partial N_i}{\partial y} \cdot \frac{\partial N_j}{\partial y} \right)_{\xi_m, \eta_n} \cdot |\det(\mathbf{J})|
\end{equation}

\subsection{Algorytm Obliczania}

\begin{enumerate}
    \item Inicjalizacja: $\mathbf{H} = \mathbf{0}_{4 \times 4}$
    \item Dla każdego punktu całkowania $(\xi_m, \eta_n)$:
    \begin{enumerate}
        \item Oblicz macierz Jakobianu $\mathbf{J}$
        \item Oblicz $\det(\mathbf{J})$ i $\mathbf{J}^{-1}$
        \item Oblicz $\frac{\partial N_i}{\partial x}$ i $\frac{\partial N_i}{\partial y}$ dla $i = 1, 2, 3, 4$
        \item Dla każdej pary $(i, j)$:
        \begin{equation}
        H_{ij} \mathrel{+}= k \cdot \left( \frac{\partial N_i}{\partial x} \cdot \frac{\partial N_j}{\partial x} + \frac{\partial N_i}{\partial y} \cdot \frac{\partial N_j}{\partial y} \right) \cdot |\det(\mathbf{J})| \cdot w_m \cdot w_n
        \end{equation}
    \end{enumerate}
\end{enumerate}

\subsection{Własności Macierzy H}

\begin{itemize}
    \item Wymiar: $4 \times 4$ (dla elementu 4-węzłowego)
    \item Symetryczna: $H_{ij} = H_{ji}$
    \item Dodatnio określona
    \item Elementy diagonalne: $H_{ii} > 0$
    \item Pasmowa po agregacji do macierzy globalnej
\end{itemize}

\subsection{Przykład Macierzy H -- Interpretacja Fizyczna}

Dla elementu kwadratowego $1 \times 1$ m z $k = 25$ W/(m$\cdot$K):

\begin{equation}
\mathbf{H}^e = \begin{bmatrix}
31.13 & -16.87 & -4.17 & -16.87 \\
-16.87 & 31.13 & -16.87 & -4.17 \\
-4.17 & -16.87 & 31.13 & -16.87 \\
-16.87 & -4.17 & -16.87 & 31.13
\end{bmatrix} \text{ [W/K]}
\end{equation}

\textbf{Interpretacja elementów macierzy:}
\begin{itemize}
    \item $H_{11} = 31.13$ -- przewodność termiczna węzła 1 (odpór cieplny)
    \item $H_{12} = -16.87$ -- sprzężenie między węzłami 1 i 2 (sąsiednie)
    \item $H_{13} = -4.17$ -- sprzężenie między węzłami 1 i 3 (przekątna)
    \item Wartości ujemne oznaczają przepływ ciepła między węzłami
    \item Suma wiersza $\approx 0$ (zachowanie bilansu energii dla izolowanego elementu)
\end{itemize}

\textbf{Analiza macierzy:}
\begin{itemize}
    \item \textbf{Dominacja diagonalna:} $|H_{ii}| > \sum_{j \neq i} |H_{ij}|$ -- układ stabilny
    \item \textbf{Symetria:} $H_{12} = H_{21}$ -- zasada wzajemności (prawo Maxwella)
    \item \textbf{Elementy pozadiagonalne ujemne:} ciepło płynie od wyższej do niższej temperatury
    \item \textbf{Wielkość elementów:} większe $|H_{ij}|$ dla węzłów bliższych (silniejsze sprzężenie)
\end{itemize}

\textbf{Pochodzenie:} Macierz $\mathbf{H}$ wynika z zastosowania metody Galerkina do równania Fouriera przewodzenia ciepła:
\begin{equation}
-\nabla \cdot (k \nabla T) = 0
\end{equation}

\section{Macierz Pojemności Cieplnej C (Capacity Matrix)}

\subsection{Teoria}

Macierz $\mathbf{C}$ reprezentuje zdolność elementu do magazynowania energii cieplnej. Jest niezbędna w analizie niestacjonarnej.

\subsection{Wzór Całkowy}

\begin{equation}
C_{ij} = \int\int_{\Omega} \rho \cdot c \cdot N_i \cdot N_j \, d\Omega
\end{equation}

gdzie:
\begin{itemize}
    \item $\rho$ -- gęstość materiału [kg/m$^3$]
    \item $c$ -- ciepło właściwe [J/(kg$\cdot$K)]
    \item $N_i, N_j$ -- funkcje kształtu
\end{itemize}

\subsection{Przykład Analityczny -- Macierz C}

Dla tego samego elementu kwadratowego $1 \times 1$ m, przy $\rho = 7800$ kg/m$^3$ i $c = 700$ J/(kg$\cdot$K):

W punkcie całkowania $(\xi, \eta) = (-1/\sqrt{3}, -1/\sqrt{3})$:

Wartości funkcji kształtu:
\begin{align}
N_1 &= 0.25(1-\xi)(1-\eta) = 0.4226 \\
N_2 &= 0.25(1+\xi)(1-\eta) = 0.0774 \\
N_3 &= 0.25(1+\xi)(1+\eta) = 0.0142 \\
N_4 &= 0.25(1-\xi)(1+\eta) = 0.0774
\end{align}

Wkład do $C_{11}$:
\begin{equation}
C_{11} \mathrel{+}= 7800 \cdot 700 \cdot (0.4226)^2 \cdot 0.25 \cdot 1.0 = 242,693 \text{ J/K}
\end{equation}

\subsection{Aproksymacja Kwadraturą Gaussa}

\begin{equation}
C_{ij} \approx \sum_{m=1}^{n_\xi} \sum_{n=1}^{n_\eta} w_m \cdot w_n \cdot \rho \cdot c \cdot N_i(\xi_m, \eta_n) \cdot N_j(\xi_m, \eta_n) \cdot |\det(\mathbf{J})|
\end{equation}

\subsection{Algorytm Obliczania}

\begin{enumerate}
    \item Inicjalizacja: $\mathbf{C} = \mathbf{0}_{4 \times 4}$
    \item Dla każdego punktu całkowania $(\xi_m, \eta_n)$:
    \begin{enumerate}
        \item Oblicz funkcje kształtu $N_i(\xi_m, \eta_n)$ dla $i = 1, 2, 3, 4$
        \item Oblicz $\det(\mathbf{J})$
        \item Dla każdej pary $(i, j)$:
        \begin{equation}
        C_{ij} \mathrel{+}= \rho \cdot c \cdot N_i \cdot N_j \cdot |\det(\mathbf{J})| \cdot w_m \cdot w_n
        \end{equation}
    \end{enumerate}
\end{enumerate}

\subsection{Własności Macierzy C}

\begin{itemize}
    \item Wymiar: $4 \times 4$
    \item Symetryczna: $C_{ij} = C_{ji}$
    \item Dodatnio określona
    \item Zazwyczaj pełna (wszystkie elementy $\neq 0$)
    \item Proporcjonalna do $\rho \cdot c$
\end{itemize}

\subsection{Przykład Macierzy C -- Interpretacja Fizyczna}

Dla elementu kwadratowego $1 \times 1$ m z $\rho = 7800$ kg/m$^3$ i $c = 700$ J/(kg$\cdot$K):

\begin{equation}
\mathbf{C}^e = \begin{bmatrix}
970,556 & 485,278 & 242,639 & 485,278 \\
485,278 & 970,556 & 485,278 & 242,639 \\
242,639 & 485,278 & 970,556 & 485,278 \\
485,278 & 242,639 & 485,278 & 970,556
\end{bmatrix} \text{ [J/K]}
\end{equation}

\textbf{Interpretacja elementów macierzy:}
\begin{itemize}
    \item $C_{11} = 970,556$ J/K -- pojemność cieplna związana z węzłem 1
    \item $C_{12} = 485,278$ J/K -- sprzężona pojemność między węzłami 1 i 2
    \item Wszystkie wartości dodatnie -- energia jest magazynowana
    \item Suma wiersza = całkowita pojemność cieplna elementu/4
    \item Element diagonalny $\approx 2 \times$ element pozadiagonalny (dla sąsiadów)
\end{itemize}

\textbf{Analiza macierzy:}
\begin{itemize}
    \item \textbf{Masa elementu:} $m = \rho \cdot V = 7800 \cdot 1 = 7800$ kg
    \item \textbf{Całkowita pojemność:} $C_{total} = m \cdot c = 7800 \cdot 700 = 5,460,000$ J/K
    \item \textbf{Suma elementów macierzy:} $\sum C_{ij} = 5,460,000$ J/K $\checkmark$
    \item \textbf{Konsystentna macierz mas:} rozkład pojemności zgodny z funkcjami kształtu
    \item \textbf{Wpływ na dynamikę:} większe $C$ $\rightarrow$ wolniejsza reakcja termiczna
\end{itemize}

\textbf{Porównanie z macierzą skupioną (lumped mass):}

Alternatywnie można użyć macierzy diagonalnej (szybsza, mniej dokładna):
\begin{equation}
\mathbf{C}^{lumped} = \begin{bmatrix}
1,365,000 & 0 & 0 & 0 \\
0 & 1,365,000 & 0 & 0 \\
0 & 0 & 1,365,000 & 0 \\
0 & 0 & 0 & 1,365,000
\end{bmatrix} \text{ [J/K]}
\end{equation}

gdzie każdy węzeł otrzymuje 1/4 całkowitej pojemności elementu.

\textbf{Pochodzenie:} Macierz $\mathbf{C}$ wynika z równania bilansu energii:
\begin{equation}
\rho \cdot c \cdot \frac{\partial T}{\partial t} = \nabla \cdot (k \nabla T) + Q
\end{equation}

\section{Warunki Brzegowe Trzeciego Rodzaju}

\subsection{Teoria Konwekcji}

Warunki brzegowe trzeciego rodzaju (warunek Robina) opisują konwekcję na brzegu obszaru:

\begin{equation}
q = \alpha (T - T_\infty)
\end{equation}

gdzie:
\begin{itemize}
    \item $q$ -- strumień ciepła [W/m$^2$]
    \item $\alpha$ -- współczynnik konwekcji [W/(m$^2 \cdot$K)]
    \item $T$ -- temperatura powierzchni [K]
    \item $T_\infty$ -- temperatura otoczenia [K]
\end{itemize}

\section{Macierz Hbc (Boundary Convection Matrix)}

\subsection{Wzór Całkowy}

Dla krawędzi brzegowej $\Gamma$:

\begin{equation}
H^{bc}_{ij} = \int_{\Gamma} \alpha \cdot N_i \cdot N_j \, d\Gamma
\end{equation}

\subsection{Aproksymacja dla Krawędzi}

Całkowanie odbywa się wzdłuż krawędzi elementu (1D):

\begin{equation}
H^{bc}_{ij} \approx \sum_{k=1}^{n} w_k \cdot \alpha \cdot N_i(\xi_k) \cdot N_j(\xi_k) \cdot \left|\frac{d\Gamma}{d\xi}\right|
\end{equation}

\subsection{Jakobian dla Krawędzi}

Długość elementu krawędzi obliczana jest jako:

\textbf{Dla krawędzi poziomych} ($\eta = \text{const}$):
\begin{equation}
\left|\frac{d\Gamma}{d\eta}\right| = \sqrt{\left(\frac{\partial x}{\partial \eta}\right)^2 + \left(\frac{\partial y}{\partial \eta}\right)^2}
\end{equation}

\textbf{Dla krawędzi pionowych} ($\xi = \text{const}$):
\begin{equation}
\left|\frac{d\Gamma}{d\xi}\right| = \sqrt{\left(\frac{\partial x}{\partial \xi}\right)^2 + \left(\frac{\partial y}{\partial \xi}\right)^2}
\end{equation}

\subsection{Definicje Krawędzi Elementu}

\begin{itemize}
    \item Krawędź 0 (dół): węzły 1--2, $\eta = -1$
    \item Krawędź 1 (prawo): węzły 2--3, $\xi = 1$
    \item Krawędź 2 (góra): węzły 3--4, $\eta = 1$
    \item Krawędź 3 (lewo): węzły 4--1, $\xi = -1$
\end{itemize}

\subsection{Algorytm Obliczania}

\begin{enumerate}
    \item Inicjalizacja: $\mathbf{H}^{bc} = \mathbf{0}_{4 \times 4}$
    \item Dla każdej krawędzi brzegowej $e = 0, 1, 2, 3$:
    \begin{enumerate}
        \item Ustal stałą współrzędną ($\xi$ lub $\eta$) i zmienną
        \item Dla każdego punktu całkowania $k$ wzdłuż krawędzi:
        \begin{enumerate}
            \item Oblicz $N_i(\xi_k, \eta_k)$ dla $i = 1, 2, 3, 4$
            \item Oblicz jakobian krawędzi $|\frac{d\Gamma}{d\xi}|$
            \item Dla każdej pary $(i, j)$:
            \begin{equation}
            H^{bc}_{ij} \mathrel{+}= \alpha \cdot N_i \cdot N_j \cdot \left|\frac{d\Gamma}{d\xi}\right| \cdot w_k
            \end{equation}
        \end{enumerate}
    \end{enumerate}
\end{enumerate}

\subsection{Własności Macierzy Hbc}

\begin{itemize}
    \item Wymiar: $4 \times 4$
    \item Symetryczna: $H^{bc}_{ij} = H^{bc}_{ji}$
    \item Rzadka (wiele zer)
    \item Niezerowe elementy tylko dla węzłów na krawędzi brzegowej
    \item Jeśli element nie ma brzegu: $\mathbf{H}^{bc} = \mathbf{0}$
\end{itemize}

\subsection{Przykład Macierzy Hbc -- Konwekcja na Dolnej Krawędzi}

Dla elementu z konwekcją na dolnej krawędzi (węzły 1--2), przy $\alpha = 300$ W/(m$^2\cdot$K), długość krawędzi $L = 1$ m:

\textbf{Obliczenie dla 2-punktowej kwadratury Gaussa:}

W punkcie $\xi = -1/\sqrt{3}$ na dolnej krawędzi ($\eta = -1$):
\begin{align}
N_1(\xi, -1) &= 0.25(1-\xi)(1-(-1)) = 0.3943 \\
N_2(\xi, -1) &= 0.25(1+\xi)(1-(-1)) = 0.1057 \\
N_3(\xi, -1) &= 0.25(1+\xi)(1+(-1)) = 0 \\
N_4(\xi, -1) &= 0.25(1-\xi)(1+(-1)) = 0
\end{align}

Jakobian krawędzi: $|d\Gamma/d\xi| = L/2 = 0.5$ m

Wkład do $H^{bc}_{11}$:
\begin{equation}
H^{bc}_{11} \mathrel{+}= 300 \cdot (0.3943)^2 \cdot 0.5 \cdot 1.0 = 23.32 \text{ W/K}
\end{equation}

\textbf{Pełna macierz Hbc dla dolnej krawędzi:}

\begin{equation}
\mathbf{H}^{bc} = \begin{bmatrix}
50.0 & 25.0 & 0 & 0 \\
25.0 & 50.0 & 0 & 0 \\
0 & 0 & 0 & 0 \\
0 & 0 & 0 & 0
\end{bmatrix} \text{ [W/K]}
\end{equation}

\textbf{Interpretacja:}
\begin{itemize}
    \item $H^{bc}_{11}, H^{bc}_{22} = 50$ W/K -- odpór konwekcyjny węzłów 1 i 2
    \item $H^{bc}_{12} = 25$ W/K -- sprzężenie między węzłami na brzegu
    \item $H^{bc}_{33}, H^{bc}_{44} = 0$ -- węzły 3 i 4 nie mają kontaktu z otoczeniem
    \item Suma rzędu 1: $50 + 25 = 75$ W/K -- całkowity odpór konwekcyjny węzła 1
    \item Większe $\alpha$ $\rightarrow$ większe elementy $H^{bc}$ $\rightarrow$ silniejsza konwekcja
\end{itemize}

\textbf{Analiza wpływu na układ:}
\begin{itemize}
    \item Macierz $\mathbf{H}^{bc}$ zwiększa sztywność układu na brzegu
    \item Modyfikuje warunki brzegowe: naturalne (Neumann) $\rightarrow$ mieszane (Robin)
    \item Stabilizuje rozwiązanie numeryczne
    \item Reprezentuje fizyczny odpór przenoszenia ciepła do otoczenia
\end{itemize}

\textbf{Pochodzenie:} Macierz $\mathbf{H}^{bc}$ wynika z zastosowania metody ważonych residuów do warunku brzegowego III rodzaju (Robin).

\section{Wektor P (Heat Flux Vector)}

\subsection{Teoria}

Wektor $\mathbf{P}$ reprezentuje obciążenie termiczne od konwekcji z otoczeniem o temperaturze $T_\infty$.

\subsection{Wzór Całkowy}

\begin{equation}
P_i = \int_{\Gamma} \alpha \cdot T_\infty \cdot N_i \, d\Gamma
\end{equation}

\subsection{Aproksymacja}

\begin{equation}
P_i \approx \sum_{k=1}^{n} w_k \cdot \alpha \cdot T_\infty \cdot N_i(\xi_k) \cdot \left|\frac{d\Gamma}{d\xi}\right|
\end{equation}

\subsection{Algorytm Obliczania}

\begin{enumerate}
    \item Inicjalizacja: $\mathbf{P} = \mathbf{0}_{4 \times 1}$
    \item Dla każdej krawędzi brzegowej:
    \begin{enumerate}
        \item Dla każdego punktu całkowania $k$:
        \begin{enumerate}
            \item Oblicz $N_i(\xi_k)$ dla $i = 1, 2, 3, 4$
            \item Oblicz jakobian krawędzi
            \item Dla każdego węzła $i$:
            \begin{equation}
            P_i \mathrel{+}= \alpha \cdot T_\infty \cdot N_i \cdot \left|\frac{d\Gamma}{d\xi}\right| \cdot w_k
            \end{equation}
        \end{enumerate}
    \end{enumerate}
\end{enumerate}

\subsection{Własności Wektora P}

\begin{itemize}
    \item Wymiar: $4 \times 1$
    \item Niezerowe tylko dla węzłów na brzegu
    \item Proporcjonalny do $\alpha \cdot T_\infty$
\end{itemize}

\subsection{Przykład Wektora P -- Obciążenie Konwekcyjne}

Dla tego samego elementu z konwekcją na dolnej krawędzi, przy $\alpha = 300$ W/(m$^2\cdot$K), $T_\infty = 1200$ K:

\textbf{Obliczenie:}

W punkcie $\xi = -1/\sqrt{3}$ na dolnej krawędzi:
\begin{equation}
P_1 \mathrel{+}= 300 \cdot 1200 \cdot 0.3943 \cdot 0.5 \cdot 1.0 = 71,074 \text{ W}
\end{equation}

\textbf{Pełny wektor P dla dolnej krawędzi:}

\begin{equation}
\mathbf{P}^e = \begin{bmatrix}
180,000 \\
180,000 \\
0 \\
0
\end{bmatrix} \text{ [W]}
\end{equation}

\textbf{Interpretacja fizyczna:}
\begin{itemize}
    \item $P_1 = P_2 = 180,000$ W -- moc cieplna docierająca do węzłów 1 i 2 z otoczenia
    \item Całkowita moc: $P_1 + P_2 = 360,000$ W
    \item Teoretyczna moc dla płaskiej ściany: $q_{total} = \alpha \cdot A \cdot (T_\infty - T_s)$
    \item Dla $\Delta T = 100$ K: $q = 300 \cdot 1 \cdot 100 = 30,000$ W
    \item Wektor P reprezentuje źródło ciepła od konwekcji
\end{itemize}

\textbf{Bilans energetyczny z Hbc i P:}

Równanie brzegowe: $\mathbf{H}^{bc} \{T\} = \{P\}$ (gdy wszystkie węzły mają $T = T_\infty$)

Sprawdzenie dla $T_1 = T_2 = 1200$ K:
\begin{equation}
\begin{bmatrix}
50 & 25 \\
25 & 50
\end{bmatrix}
\begin{bmatrix}
1200 \\
1200
\end{bmatrix}
=
\begin{bmatrix}
90,000 \\
90,000
\end{bmatrix} \text{ W}
\end{equation}

Jednak $\{P\} = [180,000, 180,000]^T$, więc:
\begin{equation}
\mathbf{H}^{bc}\{T\} - \{P\} = 
\begin{bmatrix}
90,000 - 180,000 \\
90,000 - 180,000
\end{bmatrix}
=
\begin{bmatrix}
-90,000 \\
-90,000
\end{bmatrix}
\end{equation}

Wartości ujemne oznaczają, że ciepło wpływa z otoczenia (dla $T < T_\infty$).

\textbf{Pochodzenie:} Wektor $\mathbf{P}$ wynika z członu źródłowego w warunku brzegowym konwekcyjnym.

\section{Agregacja Macierzy Lokalnych do Globalnych}

\subsection{Proces Agregacji}

Macierze lokalne (element-level) są agregowane do macierzy globalnych (system-level) według następującego algorytmu:

\begin{enumerate}
    \item Dla każdego elementu $e$:
    \begin{enumerate}
        \item Oblicz lokalne macierze: $\mathbf{H}^e$, $\mathbf{C}^e$, $\mathbf{H}^{bc,e}$, $\mathbf{P}^e$
        \item Pobierz mapowanie węzłów lokalnych $\rightarrow$ globalne: $\text{nodeIds}[i]$
        \item Dodaj wkład lokalny do macierzy globalnych
    \end{enumerate}
\end{enumerate}

\subsection{Formuła Agregacji}

Mapowanie indeksów z lokalnego do globalnego układu:

\begin{equation}
i_{\text{local}} \in [0, 3] \quad \rightarrow \quad i_{\text{global}} = \text{nodeIds}[i_{\text{local}}] - 1
\end{equation}

Dodawanie wkładów lokalnych:

\begin{align}
H_{\text{global}}[i_g][j_g] &\mathrel{+}= H^e_{\text{local}}[i][j] \\
H^{bc}_{\text{global}}[i_g][j_g] &\mathrel{+}= H^{bc,e}_{\text{local}}[i][j] \\
C_{\text{global}}[i_g][j_g] &\mathrel{+}= C^e_{\text{local}}[i][j] \\
P_{\text{global}}[i_g] &\mathrel{+}= P^e_{\text{local}}[i]
\end{align}

\subsection{Algorytm Implementacyjny}

\begin{verbatim}
dla każdego elementu e:
    oblicz: H_local, C_local, Hbc_local, P_local
    
    dla i = 0 do 3:
        globalI = nodeIds[i] - 1
        P_global[globalI] += P_local[i]
        
        dla j = 0 do 3:
            globalJ = nodeIds[j] - 1
            H_global[globalI][globalJ] += H_local[i][j]
            Hbc_global[globalI][globalJ] += Hbc_local[i][j]
            C_global[globalI][globalJ] += C_local[i][j]
\end{verbatim}

\subsection{Własności Macierzy Globalnych}

\begin{itemize}
    \item Wymiar: $N \times N$, gdzie $N$ = liczba węzłów w całej siatce
    \item Symetryczne: $\mathbf{H}_{\text{global}}$, $\mathbf{H}^{bc}_{\text{global}}$, $\mathbf{C}_{\text{global}}$
    \item Pasmowe (rzadkie) -- niezerowe tylko dla węzłów połączonych
    \item Szerokość pasma zależy od numeracji węzłów
\end{itemize}

\section{Analiza Stacjonarna (Steady-State)}

\subsection{Równanie Stanu Ustalonego}

W stanie ustalonym temperatura nie zmienia się w czasie ($\frac{\partial T}{\partial t} = 0$), więc:

\begin{equation}
[\mathbf{H}_{\text{global}} + \mathbf{H}^{bc}_{\text{global}}] \{T\} = \{P_{\text{global}}\}
\end{equation}

\textbf{Interpretacja fizyczna:}
\begin{itemize}
    \item $\mathbf{H}_{\text{global}} \{T\}$ -- strumień ciepła przez przewodzenie
    \item $\mathbf{H}^{bc}_{\text{global}} \{T\}$ -- strumień ciepła przez konwekcję
    \item $\{P_{\text{global}}\}$ -- źródła ciepła (z warunków brzegowych)
\end{itemize}

\subsection{Metoda Rozwiązania}

Układ równań liniowych rozwiązywany jest metodą eliminacji Gaussa z wyborem elementu głównego (partial pivoting):

\begin{enumerate}
    \item \textbf{Utworzenie macierzy rozszerzonej:}
    \begin{equation}
    \mathbf{A} = [\mathbf{H}_{\text{global}} + \mathbf{H}^{bc}_{\text{global}} \, | \, \mathbf{P}_{\text{global}}]
    \end{equation}
    
    \item \textbf{Eliminacja w przód z pivotingiem:}
    \begin{itemize}
        \item Dla $k = 1$ do $N-1$:
        \begin{itemize}
            \item Znajdź wiersz z maksymalnym $|A_{ik}|$ dla $i \geq k$
            \item Zamień wiersze (pivoting)
            \item Wyeliminuj $A_{ik}$ dla $i > k$
        \end{itemize}
    \end{itemize}
    
    \item \textbf{Podstawienie wstecz:}
    \begin{equation}
    T_i = \frac{1}{A_{ii}} \left( A_{i,N+1} - \sum_{j=i+1}^{N} A_{ij} \cdot T_j \right)
    \end{equation}
\end{enumerate}

\subsection{Złożoność Obliczeniowa}

\begin{itemize}
    \item Eliminacja Gaussa: $O(N^3)$
    \item Podstawienie wstecz: $O(N^2)$
    \item Całkowita: $O(N^3)$
\end{itemize}

\section{Analiza Niestacjonarna (Transient)}

\subsection{Równanie Różniczkowe}

Równanie przewodzenia ciepła zależne od czasu:

\begin{equation}
[\mathbf{C}_{\text{global}}] \left\{\frac{\partial T}{\partial t}\right\} + [\mathbf{H}_{\text{global}} + \mathbf{H}^{bc}_{\text{global}}] \{T\} = \{P_{\text{global}}\}
\end{equation}

\textbf{Interpretacja fizyczna:}
\begin{itemize}
    \item $\mathbf{C} \frac{\partial T}{\partial t}$ -- szybkość zmian energii wewnętrznej
    \item $(\mathbf{H} + \mathbf{H}^{bc}) \{T\}$ -- strumień ciepła (przewodnictwo + konwekcja)
    \item $\{P\}$ -- źródła ciepła
\end{itemize}

\subsection{Dyskretyzacja Czasowa -- Metoda Eulera Wstecznego}

Zastosowano schemat niejawny (implicit) Eulera wstecznego:

\begin{equation}
\mathbf{C} \cdot \frac{T^{n+1} - T^n}{\Delta t} + (\mathbf{H} + \mathbf{H}^{bc}) \cdot T^{n+1} = \mathbf{P}
\end{equation}

Przekształcenie do postaci liniowej:

\begin{equation}
\left[\frac{\mathbf{C}}{\Delta t} + \mathbf{H} + \mathbf{H}^{bc}\right] \{T^{n+1}\} = \left[\frac{\mathbf{C}}{\Delta t}\right] \{T^n\} + \{\mathbf{P}\}
\end{equation}

\subsection{Algorytm Krokowania Czasowego}

\begin{enumerate}
    \item \textbf{Inicjalizacja:}
    \begin{equation}
    T^0 = T_{\text{initial}} \quad \text{(dla wszystkich węzłów)}
    \end{equation}
    
    \item \textbf{Pre-obliczenia (jednorazowo):}
    \begin{equation}
    \mathbf{LHS} = \frac{\mathbf{C}}{\Delta t} + \mathbf{H} + \mathbf{H}^{bc}
    \end{equation}
    
    \item \textbf{Dla każdego kroku czasowego} $n = 1, 2, \ldots, N_{\text{steps}}$:
    \begin{enumerate}
        \item Oblicz prawą stronę:
        \begin{equation}
        \mathbf{RHS} = \frac{\mathbf{C}}{\Delta t} \cdot T^n + \mathbf{P}
        \end{equation}
        
        \item Rozwiąż układ liniowy:
        \begin{equation}
        \mathbf{LHS} \cdot T^{n+1} = \mathbf{RHS}
        \end{equation}
        
        \item Aktualizuj rozwiązanie:
        \begin{equation}
        T^n \leftarrow T^{n+1}
        \end{equation}
        
        \item Zapisz wyniki do historii
    \end{enumerate}
\end{enumerate}

\subsection{Własności Metody Eulera Wstecznego}

\begin{itemize}
    \item \textbf{Bezwarunkowo stabilna} -- dowolny krok czasowy $\Delta t$ daje stabilne rozwiązanie
    \item \textbf{Pierwszego rzędu dokładności:}
    \begin{itemize}
        \item Błąd lokalny: $O(\Delta t^2)$
        \item Błąd globalny: $O(\Delta t)$
    \end{itemize}
    \item \textbf{Niejawna (implicit)} -- wymaga rozwiązania układu równań w każdym kroku
    \item \textbf{Tłumi oscylacje} -- numeryczna dyfuzja wygładza rozwiązanie
\end{itemize}

\subsection{Stabilność}

Metoda Eulera wstecznego jest \textbf{A-stabilna}, co oznacza, że dla dowolnego $\Delta t > 0$:

\begin{equation}
\lim_{n \to \infty} T^n = T_{\text{steady-state}}
\end{equation}

Brak ograniczenia na krok czasowy ze względów stabilności (może być ograniczony tylko dokładnością).

\subsection{Rola Macierzy C}

\begin{itemize}
    \item \textbf{Bez C:} reakcja natychmiastowa $\rightarrow$ problem stacjonarny
    \item \textbf{Z C:} opóźniona reakcja, akumulacja energii
    \item \textbf{Większe C} $\rightarrow$ wolniejsze zmiany temperatury
    \item C stabilizuje i symetryzuje problem niestacjonarny
\end{itemize}

\subsection{Liczba Kroków Czasowych}

\begin{equation}
N_{\text{steps}} = \frac{t_{\text{simulation}}}{\Delta t}
\end{equation}

\textbf{Dobór kroku czasowego:}
\begin{itemize}
    \item Mały $\Delta t$ -- wyższa dokładność, więcej obliczeń
    \item Duży $\Delta t$ -- niższa dokładność, mniej obliczeń
    \item Kompromis między dokładnością a efektywnością
\end{itemize}

\section{Parametry Materiałowe i ich Wpływ}

\subsection{Parametry Wejściowe}

\begin{table}[H]
\centering
\begin{tabular}{|l|c|l|c|}
\hline
\textbf{Parametr} & \textbf{Symbol} & \textbf{Jednostka} & \textbf{Wpływ na} \\
\hline
Przewodność cieplna & $k$ & W/(m·K) & $\mathbf{H}$ \\
\hline
Współczynnik konwekcji & $\alpha$ & W/(m$^2$·K) & $\mathbf{H}^{bc}$, $\mathbf{P}$ \\
\hline
Temp. otoczenia & $T_\infty$ & K lub °C & $\mathbf{P}$ \\
\hline
Gęstość & $\rho$ & kg/m$^3$ & $\mathbf{C}$ \\
\hline
Ciepło właściwe & $c$ & J/(kg·K) & $\mathbf{C}$ \\
\hline
Temp. początkowa & $T_0$ & K lub °C & Warunek początkowy \\
\hline
\end{tabular}
\caption{Parametry materiałowe i ich wpływ na macierze}
\end{table}

\subsection{Wpływ na Rozwiązanie}

\begin{itemize}
    \item \textbf{Zwiększenie $k$:} szybsza dyfuzja ciepła, większa macierz $\mathbf{H}$
    \item \textbf{Zwiększenie $\alpha$:} silniejsza konwekcja, większa $\mathbf{H}^{bc}$ i $\mathbf{P}$
    \item \textbf{Zwiększenie $\rho$ lub $c$:} większa pojemność cieplna, większa $\mathbf{C}$, wolniejsze zmiany temperatury
    \item \textbf{Zwiększenie $T_\infty$:} większe obciążenie termiczne, większy $\mathbf{P}$
\end{itemize}

\section{Kolejność Obliczeń w Programie}

\subsection{Etap 1: Przygotowanie Danych}

\begin{enumerate}
    \item Wczytanie siatki z pliku (węzły, elementy)
    \item Wczytanie parametrów materiałowych
    \item Wczytanie warunków brzegowych
    \item Zainicjalizowanie struktur danych
\end{enumerate}

\subsection{Etap 2: Obliczenia Lokalne (dla każdego elementu)}

\begin{enumerate}
    \item Pobranie współrzędnych węzłów elementu: $(x_i, y_i)$
    \item \textbf{Punkty całkowania Gaussa} (stałe): $\xi_k, \eta_k, w_k$
    \item Dla każdego punktu całkowania $(\xi_m, \eta_n)$:
    \begin{enumerate}
        \item Oblicz \textbf{funkcje kształtu}: $N_i(\xi, \eta)$
        \item Oblicz \textbf{pochodne funkcji kształtu}: $\frac{\partial N_i}{\partial \xi}$, $\frac{\partial N_i}{\partial \eta}$
        \item Oblicz \textbf{macierz Jakobianu}: $\mathbf{J}$
        \item Oblicz \textbf{wyznacznik Jakobianu}: $\det(\mathbf{J})$
        \item Oblicz \textbf{odwrotną macierz Jakobianu}: $\mathbf{J}^{-1}$
        \item Oblicz \textbf{pochodne fizyczne}: $\frac{\partial N_i}{\partial x}$, $\frac{\partial N_i}{\partial y}$
        \item Akumuluj wkład do \textbf{macierzy lokalnej H}
        \item Akumuluj wkład do \textbf{macierzy lokalnej C}
    \end{enumerate}
    \item Dla każdej krawędzi brzegowej:
    \begin{enumerate}
        \item Oblicz \textbf{jakobian krawędzi}: $|\frac{d\Gamma}{d\xi}|$
        \item Akumuluj wkład do \textbf{macierzy lokalnej Hbc}
        \item Akumuluj wkład do \textbf{wektora lokalnego P}
    \end{enumerate}
\end{enumerate}

\subsection{Etap 3: Agregacja Globalna}

\begin{enumerate}
    \item Dla każdego elementu:
    \begin{enumerate}
        \item Mapowanie węzłów lokalnych $\rightarrow$ globalnych
        \item Dodanie $\mathbf{H}^e$ do $\mathbf{H}_{\text{global}}$
        \item Dodanie $\mathbf{C}^e$ do $\mathbf{C}_{\text{global}}$
        \item Dodanie $\mathbf{H}^{bc,e}$ do $\mathbf{H}^{bc}_{\text{global}}$
        \item Dodanie $\mathbf{P}^e$ do $\mathbf{P}_{\text{global}}$
    \end{enumerate}
\end{enumerate}

\subsection{Etap 4: Rozwiązanie}

\textbf{Analiza stacjonarna (opcjonalnie):}
\begin{enumerate}
    \item Utworzenie macierzy rozszerzonej: $[\mathbf{H} + \mathbf{H}^{bc} \, | \, \mathbf{P}]$
    \item Eliminacja Gaussa z pivotingiem
    \item Podstawienie wstecz
    \item Wyświetlenie rozkładu temperatury
\end{enumerate}

\textbf{Analiza niestacjonarna:}
\begin{enumerate}
    \item Inicjalizacja: $T^0 = T_{\text{initial}}$
    \item Pre-obliczenia: $\mathbf{LHS} = \frac{\mathbf{C}}{\Delta t} + \mathbf{H} + \mathbf{H}^{bc}$
    \item Dla każdego kroku czasowego:
    \begin{enumerate}
        \item Oblicz $\mathbf{RHS} = \frac{\mathbf{C}}{\Delta t} \cdot T^n + \mathbf{P}$
        \item Rozwiąż $\mathbf{LHS} \cdot T^{n+1} = \mathbf{RHS}$
        \item Aktualizuj $T^n \leftarrow T^{n+1}$
        \item Zapisz historię
    \end{enumerate}
    \item Wyświetlenie wyników końcowych i historii
\end{enumerate}

\section{Przykład Kompletny -- Analiza Prostego Układu}

\subsection{Definicja Problemu}

Rozważmy płytę stalową o wymiarach $0.1 \times 0.1$ m, grubość zaniedbywalnie mała (analiza 2D). Dolna krawędź ma kontakt z gorącym otoczeniem ($T_\infty = 1200°$C, $\alpha = 300$ W/(m$^2\cdot$K)), pozostałe krawędzie izolowane. Temperatura początkowa $T_0 = 100°$C.

\textbf{Dane materiałowe (stal):}
\begin{itemize}
    \item Przewodność: $k = 25$ W/(m$\cdot$K)
    \item Gęstość: $\rho = 7800$ kg/m$^3$
    \item Ciepło właściwe: $c = 700$ J/(kg$\cdot$K)
\end{itemize}

\subsection{Dyskretyzacja -- Jeden Element}

Najprostsza siatka: 1 element, 4 węzły:
\begin{itemize}
    \item Węzeł 1: $(0, 0)$ -- dolny lewy (brzeg)
    \item Węzeł 2: $(0.1, 0)$ -- dolny prawy (brzeg)
    \item Węzeł 3: $(0.1, 0.1)$ -- górny prawy
    \item Węzeł 4: $(0, 0.1)$ -- górny lewy
\end{itemize}

\subsection{Macierze Elementowe}

\textbf{Macierz H} (przeskalowana dla $0.1 \times 0.1$ m):
\begin{equation}
\mathbf{H} = \begin{bmatrix}
3.113 & -1.687 & -0.417 & -1.687 \\
-1.687 & 3.113 & -1.687 & -0.417 \\
-0.417 & -1.687 & 3.113 & -1.687 \\
-1.687 & -0.417 & -1.687 & 3.113
\end{bmatrix} \text{ [W/K]}
\end{equation}

\textbf{Macierz C:}
\begin{equation}
\mathbf{C} = \begin{bmatrix}
9.706 & 4.853 & 2.426 & 4.853 \\
4.853 & 9.706 & 4.853 & 2.426 \\
2.426 & 4.853 & 9.706 & 4.853 \\
4.853 & 2.426 & 4.853 & 9.706
\end{bmatrix} \text{ [J/K]}
\end{equation}

\textbf{Macierz Hbc} (tylko dolna krawędź):
\begin{equation}
\mathbf{H}^{bc} = \begin{bmatrix}
5.0 & 2.5 & 0 & 0 \\
2.5 & 5.0 & 0 & 0 \\
0 & 0 & 0 & 0 \\
0 & 0 & 0 & 0
\end{bmatrix} \text{ [W/K]}
\end{equation}

\textbf{Wektor P:}
\begin{equation}
\mathbf{P} = \begin{bmatrix}
18,000 \\
18,000 \\
0 \\
0
\end{bmatrix} \text{ [W]}
\end{equation}

\subsection{Rozwiązanie Stacjonarne}

Układ równań: $[\mathbf{H} + \mathbf{H}^{bc}]\{T\} = \{P\}$

\begin{equation}
\begin{bmatrix}
8.113 & -0.813 & -0.417 & -1.687 \\
-0.813 & 8.113 & -1.687 & -0.417 \\
-0.417 & -1.687 & 3.113 & -1.687 \\
-1.687 & -0.417 & -1.687 & 3.113
\end{bmatrix}
\begin{bmatrix}
T_1 \\
T_2 \\
T_3 \\
T_4
\end{bmatrix}
=
\begin{bmatrix}
18,000 \\
18,000 \\
0 \\
0
\end{bmatrix}
\end{equation}

\textbf{Rozwiązanie:}
\begin{equation}
\{T\} \approx \begin{bmatrix}
1150 \\
1150 \\
1080 \\
1080
\end{bmatrix} \text{ [°C]}
\end{equation}

\textbf{Interpretacja:}
\begin{itemize}
    \item Węzły 1, 2 (dolne): $\approx 1150°$C -- blisko temperatury otoczenia
    \item Węzły 3, 4 (górne): $\approx 1080°$C -- chłodniejsze, dalej od źródła
    \item Gradient temperatury: $\Delta T \approx 70°$C na wysokości $0.1$ m
    \item Rozkład zgodny z intuicją: najcieplej na dole, chłodniej na górze
\end{itemize}

\subsection{Analiza Niestacjonarna}

Dla kroku czasowego $\Delta t = 1$ s, równanie w pierwszym kroku:

\begin{equation}
\left[\frac{\mathbf{C}}{1} + \mathbf{H} + \mathbf{H}^{bc}\right]\{T^1\} = \frac{\mathbf{C}}{1}\{T^0\} + \{P\}
\end{equation}

Przy $T^0 = 100°$C:

\textbf{Prawa strona:}
\begin{equation}
\mathbf{RHS} = \mathbf{C}\{T^0\} + \{P\} = \begin{bmatrix}
9.706 & 4.853 & 2.426 & 4.853 \\
\vdots & \vdots & \vdots & \vdots
\end{bmatrix}
\begin{bmatrix}
100 \\
100 \\
100 \\
100
\end{bmatrix}
+
\begin{bmatrix}
18,000 \\
18,000 \\
0 \\
0
\end{bmatrix}
\approx
\begin{bmatrix}
20,184 \\
20,184 \\
2,184 \\
2,184
\end{bmatrix}
\end{equation}

Po rozwiązaniu: $T^1 \approx [650, 650, 350, 350]^T$ °C

Temperatura rośnie szybciej na dole (konwekcja) niż na górze (przewodnictwo).

\section{Weryfikacja Wyników}

\subsection{Warunki Poprawności}

\begin{itemize}
    \item $\det(\mathbf{J}) > 0$ dla wszystkich punktów całkowania (element niezdegenerowany)
    \item Macierze $\mathbf{H}$, $\mathbf{C}$, $\mathbf{H}^{bc}$ symetryczne
    \item Macierze $\mathbf{H}$, $\mathbf{C}$ dodatnio określone
    \item Bilans energii: $\sum$ (strumienie wejściowe) = $\sum$ (strumienie wyjściowe) + $\frac{dE}{dt}$
\end{itemize}

\subsection{Testy Numeryczne}

\begin{enumerate}
    \item \textbf{Test symetrii:} sprawdzenie czy $H_{ij} = H_{ji}$
    \item \textbf{Test zbieżności:} zmniejszenie $\Delta t$ powinno poprawić dokładność
    \item \textbf{Test porównawczy:} porównanie z rozwiązaniem analitycznym (jeśli dostępne)
    \item \textbf{Test bilansu energii:} całkowita energia powinna być zachowana
\end{enumerate}

\subsection{Weryfikacja Bilansu Energii}

Dla przykładu niestacjonarnego, sprawdzamy bilans w pierwszym kroku:

\textbf{Energia na początku:}
\begin{equation}
E^0 = \sum_i C_{total,i} \cdot T^0_i = (4 \times 9.706) \cdot 100 \approx 3,882 \text{ J}
\end{equation}

\textbf{Energia na końcu kroku:}
\begin{equation}
E^1 = \sum_i C_{total,i} \cdot T^1_i \approx 19,412 \text{ J}
\end{equation}

\textbf{Zmiana energii:}
\begin{equation}
\Delta E = E^1 - E^0 = 15,530 \text{ J}
\end{equation}

\textbf{Dostarczona moc $\times$ czas:}
\begin{equation}
Q_{in} = (P_1 + P_2) \cdot \Delta t = 36,000 \cdot 1 = 36,000 \text{ J}
\end{equation}

\textbf{Utrata przez przewodność wewnętrzną i konwekcję:}
\begin{equation}
Q_{loss} = Q_{in} - \Delta E = 20,470 \text{ J}
\end{equation}

Bilans jest zachowany -- część energii podnosi temperaturę ($\Delta E$), część ucieka przez przewodność.

\section{Podsumowanie}

Program implementuje pełną analizę MES dla przewodzenia ciepła, obejmując:

\begin{itemize}
    \item \textbf{Dyskretyzację przestrzenną:} elementy 4-węzłowe z funkcjami kształtu bilinearnymi
    \item \textbf{Całkowanie numeryczne:} kwadratura Gaussa 2/3/4-punktowa
    \item \textbf{Transformacje geometryczne:} macierz Jakobianu i jej odwrotność
    \item \textbf{Macierze elementowe:} $\mathbf{H}$ (przewodnictwo), $\mathbf{C}$ (pojemność), $\mathbf{H}^{bc}$ (konwekcja)
    \item \textbf{Wektor obciążenia:} $\mathbf{P}$ (źródła cieplne)
    \item \textbf{Agregację:} z lokalnych do globalnych macierzy
    \item \textbf{Analizę stacjonarną:} eliminacja Gaussa
    \item \textbf{Analizę niestacjonarną:} metoda Eulera wstecznego
\end{itemize}

Wszystkie wzory i algorytmy są wyprowadzone z podstawowych zasad mechaniki kontinuum, termodynamiki i metody elementów skończonych, zapewniając solidne podstawy teoretyczne dla implementacji numerycznej.

\end{document}
